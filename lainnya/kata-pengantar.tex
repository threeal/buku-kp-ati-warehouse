\begin{center}
  \Large\textbf{KATA PENGANTAR}
\end{center}
\vspace{2ex}

\addcontentsline{toc}{chapter}{KATA PENGANTAR}

% Pengaturan ukuran indentasi
\setlength{\parindent}{7ex}

% Ubah paragraf-paragraf berikut sesuai dengan yang ingin diisi pada kata pengantar

Puji syukur kami panjatkan kepada Allah SWT karena hanya dengan rahmat dan hidayah-Nya Penulis dapat melaksanakan kerja praktik dan menyelesaikan laporan kerja praktik di PT. Aneka Tuna Indonesia dengan topik “PEMBUATAN APLIKASI WEBSITE UNLOADING BASKET SECTION”.
Dalam pelakasanaan maupun penulisan laporan kerja praktik ini, Penulis mengucapkan terima kasih atas bantuan, arahan, dan motivasi yang diberikan baik secara langsung ataupun tidak langsung dari:
\vspace{0.5ex}

\begin{enumerate}[nolistsep]

  \item Bapak Arief Kurniawan, S.T., M.T. selaku Dosen Pembimbing di Departemen Teknik Komputer FTEIC - ITS, yang telah memberikan bimbingan kepada kami selama mengerjakan kerja praktik di perusahaan.
  \vspace{0.5ex}

  \item Bapak Jarot Kusumo Wibowo yang telah banyak memberikan arahan kepada kami.
  \vspace{0.5ex}

  \item Semua anggota PT. Aneka Tuna Indonesia yang telah memberikan ilmu-ilmu baru kepada kami serta berkenan untuk kami wawancarai maupun kami ajak berdiskusi.
  \vspace{0.5ex}

  \item Serta semua pihak yang tidak bisa disebutkan satu-persatu yang turut membantu dan memperlancar jalannya kerja praktik ini.
  \vspace{0.5ex}

\end{enumerate}
\vspace{0.5ex}

Penulis menyadari bahwa masih banyak kekurangan dalam perancangan dan pembuatan laporan Kerja Praktik ini.
Semoga buku laporan kerja praktik ini dapat memberikan manfaat bagi para pembaca, khususnya bagi penulis sendiri.
\vspace{2ex}

\begin{flushright}
  \begin{tabular}[b]{c}
    % Ubah kalimat berikut sesuai dengan tempat, bulan, dan tahun penulisan
    Surabaya, 12 Oktober 2020
    \\
    \\
    \\
    \\
    Penulis
  \end{tabular}
\end{flushright}
