% Ubah kalimat sesuai dengan judul dari bab ini
\chapter{KESIMPULAN DAN SARAN}
\vspace{4ex}

% Pengaturan ukuran indentasi
\setlength{\parindent}{7ex}

% Ubah konten-konten berikut sesuai dengan yang ingin diisi pada bab ini

\section{Kesimpulan}
\vspace{1ex}

Kesimpulan yang kami peroleh dari hasil kerja praktik ini, antara lain:
\vspace{0.5ex}

\begin{enumerate}[nolistsep]

  \item Penggunaan database yang bersifat digital dapat digunakan untuk menggantikan sistem administrasi loading barang sebelumnya yang berbasis kertas.
  \vspace{0.5ex}

  \item Data administrasi yang ditaruh di server dapat lebih mudah untuk diakses, ditambah, diubah, dan dihapus dibandingkan dengan data yang hanya berbasis kertas.
  \vspace{0.5ex}

  \item \emph{Progressive Web Apps} dapat digunakan untuk mengubah website menjadi aplikasi sehingga sistem yang dibuat bisa dengan mudah diakses pada perangkat mobile terutama jika dibutuhkan mobilitas dan kemudahan untuk melakukan pencatatan di lapangan.
  \vspace{0.5ex}

  \item Data administrasi yang ada pada server dapat diunduh dalam bentuk \emph{spreadsheet} yang bisa dicetak sehingga mempermudah pelaporan data administrasi yang masih membutuhkan bentuk kertas.
  \vspace{0.5ex}

\end{enumerate}
\vspace{0.5ex}

\section{Saran}
\vspace{1ex}

Penulis menyadari pentingnya keberadaan sistem baru yang telah dibuat ini, namun penulis menemukan beberapa hal yang kami rasa perlu untuk diperbaiki dan ditingkatkan, antara lain:
\vspace{0.5ex}

\begin{enumerate}[nolistsep]

  \item Perlunya mekanisme keamanan yang lebih baik serta backup pada data yang ada agar data yang berada di server tidak dengan mudah disalahgunakan oleh pihak yang tidak bertanggung jawab.
  \vspace{0.5ex}

  \item Perlunya mekanisme penyimpanan dan pengolahan data yang lebih baik agar data yang disimpan tidak memakan banyak ruang serta bisa diakses dengan lebih cepat dan optimal.
  \vspace{0.5ex}

  \item Kedepannya disarankan agar PWA yang ada bisa juga digunakan secara offline, sehingga tanpa adanya akses ke server, sistem masih bisa digunakan untuk sementara sampai akses ke server tersebut telah diperoleh kembali.
  \vspace{0.5ex}

\end{enumerate}
\vspace{0.5ex}