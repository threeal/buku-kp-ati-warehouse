% Ubah kalimat sesuai dengan judul dari bab ini
\chapter{DESAIN DAN IMPLEMENTASI}
\vspace{4ex}

% Pengaturan ukuran indentasi
\setlength{\parindent}{7ex}

% Ubah konten-konten berikut sesuai dengan yang ingin diisi pada bab ini

\section{Deskripsi Sistem}
\vspace{1ex}

Sistem yang kami buat merupakan sebuah website yang bisa digunakan untuk melakukan administrasi loading barang pada PT. Aneka Tuna Indonesia yang sebelumnya secara penuh berbasis kertas.
Sistem ini nantinya akan dibuat menggunakan \emph{Node.js} sebagai kerangka utama dari program dan \emph{MongoDB} sebagai database dari sistem tersebut yang berbasis \emph{document-oriented}.
\vspace{0.5ex}

Pada PT. Aneka Tuna Indoensia, administrasi loading barang dilakukan oleh petugas tertentu yang memiliki jadwal kerja yang berbeda.
Untuk itu diperlukan sistem akun yang bisa digunakan oleh petugas-petugas tersebut untuk melakukan pencatatan, sehingga hanya petugas-petugas tersebut dan beberapa pihak lain yang bisa mengakses data yang ada pada sistem.
Selain itu untuk mempermudah pencatatan terutama di sisi mobilitas, sistem yang kami buat juga mendukung adanya PWA sehingga website yang ada bisa dengan mudah terpasang pada perangkat mobile layaknya aplikasi pada umumnya.
\vspace{0.5ex}

\section{Spesifikasi Kasus Penggunaan}

Secara umum, kasus penggunaan dari sistem yang kami buat bisa dijabarkan sebagai berikut:
\vspace{0.5ex}

\begin{enumerate}[nolistsep]

  \item Sistem login dan pendaftaran akun baru.
  \vspace{0.5ex}

  \item Manajemen pengguna oleh admin, termasuk memverifikasi akun baru, mengangkat maupun melepas pengguna dari jabatan admin, serta penghapusan akun.
  \vspace{0.5ex}

  \item Manajemen data administrasi loading barang yang meliputi data produk, dokumen muat palet, dokumen bongkar basket, dan lain sebagainya.
  \vspace{0.5ex}

  \item Pengunduhan data administrasi loading barang dalam bentuk \emph{spreadsheet}.
  \vspace{0.5ex}

\end{enumerate}
\vspace{0.5ex}

\section{Implementasi Sistem}
\vspace{1ex}

Pada bagian ini, kami akan menjelaskan implementasi yang kami lakukan pada sistem yang kami buat dengan melihat spesifikasi kasus penggunaan seperti yang dijelaskan sebelumnya.
Untuk mempersingkat cakupan bahasan pada buku ini, kami hanya akan menampilkan sebagian kecil kode program yang kami gunakan.
Untuk Detail lebih lengkap dari kode program sistem yang kami buat bisa dilihat pada repository \emph{GitHub} \citep{repoGithub} yang ada.

\subsection{Implementasi \emph{Database}}
\vspace{1ex}

\lipsum[3]
\vspace{0.5ex}

\subsection{Implementasi \emph{REST API}}
\vspace{1ex}

\lipsum[4]
\vspace{0.5ex}

\subsection{Implementasi Aplikasi}
\vspace{1ex}

Aplikasi diimplementasikan dengan \lipsum[2]
\vspace{0.5ex}

% Digunakan untuk page break
\newpage

% Contoh pembuatan code snippet
\begin{lstlisting}[
  language=C++,
  label={lst:helloWorld},
  caption={Hello World}
]
#include <iostream>

int main() {
    std::cout << "Hello World!";
    return 0;
}
\end{lstlisting}
\vspace{0.5ex}

% Contoh penggunaan referensi dari code snippet yang diinputkan
Seperti contoh pada baris program \ref{lst:helloWorld} dan \ref{lst:bilanganPrima}, \lipsum[3]
\vspace{0.5ex}

% Contoh input code snippet
\lstinputlisting[
  % Bahasa yang digunakan oleh code snippet
  language=Python,
  % Label referensi dari code snippet yang diinputkan
  label={lst:bilanganPrima},
  % Keterangan dari code snippet yang diinputkan
  caption={Perhitungan Bilangan Prima}
% Nama dari file code snippet yang diinputkan
]{program/prime-number.py}
\vspace{0.5ex}