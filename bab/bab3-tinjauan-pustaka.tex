% Ubah kalimat sesuai dengan judul dari bab ini
\chapter{TINJAUAN PUSTAKA}
\vspace{4ex}

% Pengaturan ukuran indentasi
\setlength{\parindent}{7ex}

% Ubah konten-konten berikut sesuai dengan yang ingin diisi pada bab ini

\section{\emph{Node.js}}
\vspace{1ex}
Node.js adalah open source, cross-platform, back end, JavaScript runtime environment yang mengeksekusi kode 
JavaScript di luar web browser. 
Node.js memungkinkan pengembang menggunakan JavaScript untuk menulis alat baris 
perintah dan skrip untuk menjalankan server-side scripting untuk menghasilkan konten halaman web dinamis sebelum
halaman dikirim ke browser Web pengguna. 
Node.js menyatukan pengembangan aplikasi web di satu bahasa pemrograman.
\vspace{0.5ex}

\section{\emph{Document-oriented Database}}
\vspace{1ex}
Document-oriented database adalah program komputer yang dirancang untuk menyimpan, mengambil, dan mengelola 
informasi berorientasi dokumen.
Document-oriented database adalah salah satu kategori utama database NoSQL. 
Konsep sentral dari Document-oriented database adalah pengertian tentang dokumen. Secara umum, 
mereka semua menganggap dokumen mengenkapsulasi dan menyandikan data (atau informasi) dalam beberapa format 
standar atau pengkodean. Pengkodean yang digunakan termasuk XML, YAML, JSON, serta bentuk biner seperti BSON.
\vspace{0.5ex}

\subsection{\emph{MongoDB}}
\vspace{1ex}
MongoDB adalah program cross-platform untuk document-oriented database dan diklasifikasikan sebagai program database 
NoSQL. MongoDB menggunakan dokumen dengan skema opsional yang fleksibel seperti JSON, yang berarti bidang dapat 
bervariasi dari dokumen ke dokumen dan struktur data dapat diubah seiring waktu. MongoDB adalah basis data yang 
terdistribusi pada intinya, sehingga ketersediaan, penskalaan, dan distribusi dapat dengan mudah 
dibangun dan digunakan.
\vspace{0.5ex}

\subsection{\emph{Mongoose}}
\vspace{1ex}

\lipsum[4]
\vspace{0.5ex}

\section{\emph{Representational State Transfer API} (\emph{REST API})}
\vspace{1ex}
Representational state transfer (REST) adalah gaya arsitektur perangkat lunak yang mendefinisikan sekumpulan constrain
yang akan digunakan untuk membuat layanan Web. Layanan web yang sesuai dengan gaya arsitektur REST disebut 
layanan Web RESTful yang menyediakan interoperabilitas antara sistem komputer di internet. Layanan Web RESTful 
memungkinkan sistem yang meminta untuk mengakses dan memanipulasi representasi tekstual dari Web resources
dengan menggunakan serangkaian operasi tanpa pernyataan yang seragam dan telah ditentukan sebelumnya.

\lipsum[5]
\vspace{0.5ex}

\subsection{\emph{Express}}
\vspace{1ex}
Express.js, atau Express, adalah kerangka aplikasi web back end untuk Node.js, dirilis sebagai free open-source 
software di bawah Lisensi MIT. Express.js dirancang untuk membangun aplikasi web dan API dan merupakan 
framework server standar untuk Node.js. Express js. dideskripsikannya sebagai server yang terinspirasi dari Sinatra, 
yang berarti bahwa server ini relatif minimalis dengan banyak fitur yang tersedia sebagai plugin. Express adalah 
komponen back-end dari MEAN stack, bersama dengan perangkat lunak database MongoDB dan framework front-end AngularJS.
\vspace{0.5ex}

\subsection{\emph{Axios}}
\vspace{1ex}

\lipsum[7]
\vspace{0.5ex}

\section{\emph{Progressive Web App} (\emph{PWA})}
\vspace{1ex}
Progressive Web App (PWA) adalah jenis perangkat lunak aplikasi yang dikirimkan melalui web, dibuat menggunakan 
teknologi web umum seperti HTML, CSS, dan JavaScript untuk bekerja pada platform apa pun yang menggunakan browser 
yang sesuai standar, termasuk desktop dan perangkat seluler.
Fitur PWA memungkinkan untuk menutup celah ke aplikasi asli dan menciptakan pengalaman pengguna yang serupa seperti 
bekerja secara offline, performa yang cepat, akses ke dalam sensor ponsel, dukungan untuk push notification, dan 
ikon di layar beranda ponsel.
\vspace{0.5ex}

\subsection{\emph{Vue.js}}
\vspace{1ex}
Vue.js adalah open-source framework front-end JavaScript untuk membangun antarmuka 
pengguna dan aplikasi pada satu halaman.
Vue.js menampilkan arsitektur yang dapat disesuaikan secara bertahap yang berfokus pada rendering deklaratif dan 
komposisi komponen dengan inti Library yang difokuskan pada lapisan tampilan saja. Fitur-fitur canggih yang 
diperlukan untuk aplikasi kompleks seperti perutean, manajemen status, dan perkakas build ditawarkan melalui 
Library dan paket pendukung yang dikelola secara resmi.
Vue.js memungkinkan kita untuk memperluas HTML dengan atribut HTML yang disebut directives. Directives menawarkan 
fungsionalitas ke aplikasi HTML, dan datang sebagai bawaan atau yang ditentukan pengguna.
\vspace{0.5ex}

\subsection{\emph{Vuetify}}
\vspace{1ex}
Vuetify adalah Library antarmuka Vue dengan Komponen Material untuk memperindah tampilan. 
Tujuan Vuetify adalah menyediakan semua yang dibutuhkan pengguna untuk membangun aplikasi web yang indah dan 
menarik menggunakan spesifikasi Desain Material dan dengan siklus pembaruan yang konsisten, 
Dukungan Jangka Panjang (LTS), keterlibatan komunitas yang responsif, ekosistem sumber daya yang luas, 
dan dedikasi pada komponen berkualitas.
\vspace{0.5ex}