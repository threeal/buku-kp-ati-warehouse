% Ubah kalimat sesuai dengan judul dari bab ini
\chapter{PENDAHULUAN}
\vspace{4ex}

% Pengaturan ukuran indentasi
\setlength{\parindent}{7ex}

% Ubah konten-konten berikut sesuai dengan yang ingin diisi pada bab ini

\section{Latar Belakang}
\vspace{1ex}

Pesatnya perkembangan roket yang merupakan \lipsum[1]
\vspace{0.5ex}

\lipsum[2]
\vspace{0.5ex}

% Digunakan untuk page break
\newpage

\section{Rumusan Permasalahan}
\vspace{1ex}

Masalah yang akan \lipsum[1][1] adalah:
\vspace{0.5ex}

\begin{enumerate}[nolistsep]

  \item Bagaimana cara \lipsum[1][1-2]
  \vspace{0.5ex}

  \item \lipsum[1][3-4]
  \vspace{0.5ex}

\end{enumerate}
\vspace{0.5ex}

\section{Tujuan}
\vspace{1ex}

Tujuan dari \lipsum[1][1] adalah:
\vspace{0.5ex}

\begin{enumerate}[nolistsep]

  \item Membuat \lipsum[1][1-2]
  \vspace{0.5ex}

  \item \lipsum[1][3-4]
  \vspace{0.5ex}

\end{enumerate}
\vspace{0.5ex}

\section{Manfaat}
\vspace{1ex}

Manfaat dari \lipsum[1][1] adalah:
\vspace{0.5ex}

\begin{enumerate}[nolistsep]

  \item Mempermudah \lipsum[1][1-2]
  \vspace{0.5ex}

  \item \lipsum[1][3-4]
  \vspace{0.5ex}

\end{enumerate}
\vspace{0.5ex}

\section{Waktu dan Tempat Pelaksanaan}
\vspace{1ex}

Kerja praktik akan dilaksanakan pada \lipsum[1][1-3]
\vspace{0.5ex}

% Digunakan untuk page break
\newpage

\section{Metodologi Kerja Praktik}
\vspace{1ex}

Metode yang \lipsum[1][1] yaitu:
\vspace{0.5ex}

\begin{enumerate}[nolistsep]

  \item \textbf{Perumusan Masalah}
  \vspace{0.5ex}

  Pada tahap ini \lipsum[1][1-3]
  \vspace{0.5ex}

  \item \textbf{Studi Literatur}
  \vspace{0.5ex}

  Pada tahap ini \lipsum[1][1-3]
  \vspace{0.5ex}

  \item \textbf{Analisis dan Perancangan Sistem}
  \vspace{0.5ex}

  Pada tahap ini \lipsum[1][1-3]
  \vspace{0.5ex}

  \item \textbf{Implementasi Sistem}
  \vspace{0.5ex}

  Pada tahap ini \lipsum[1][1-3]
  \vspace{0.5ex}

  \item \textbf{Pengujian dan Evaluasi}
  \vspace{0.5ex}

  Pada tahap ini \lipsum[1][1-3]
  \vspace{0.5ex}

\end{enumerate}
\vspace{0.5ex}

\section{Sistematika Penulisan}
\vspace{1ex}

Laporan kerja praktik akan \lipsum[1][1] yaitu:
\vspace{0.5ex}

\begin{enumerate}[nolistsep]

  \item \textbf{Bab I Pendahuluan}
  \vspace{0.5ex}

  Bab ini berisi \lipsum[1][1-3]
  \vspace{0.5ex}

  \item \textbf{Bab II Profil Perusahaan}
  \vspace{0.5ex}

  Bab ini berisi \lipsum[1][1-3]
  \vspace{0.5ex}

  \item \textbf{Bab III Tinjauan Pustaka}
  \vspace{0.5ex}

  Bab ini berisi \lipsum[1][1-3]
  \vspace{0.5ex}

  \item \textbf{Bab IV Desain dan Implementasi}
  \vspace{0.5ex}

  Bab ini berisi \lipsum[1][1-3]
  \vspace{0.5ex}

  \item \textbf{Bab V Pengujian dan Evaluasi}
  \vspace{0.5ex}

  Bab ini berisi \lipsum[1][1-3]
  \vspace{0.5ex}

  \item \textbf{Bab VI Kesimpulan dan Saran}
  \vspace{0.5ex}

  Bab ini berisi \lipsum[1][1-3]
  \vspace{0.5ex}

\end{enumerate}
\vspace{0.5ex}
