% Ubah kalimat sesuai dengan judul dari bab ini
\chapter{PENDAHULUAN}
\vspace{4ex}

% Pengaturan ukuran indentasi
\setlength{\parindent}{7ex}

% Ubah konten-konten berikut sesuai dengan yang ingin diisi pada bab ini

\section{Latar Belakang}
\vspace{1ex}

Sebagai bentuk realisasi kebijaksanaan pemerintah dalam peningkatan mutu pendidikan perguruan tinggi dan untuk mendukung program link and match antara perguruan tinggi dengan dunia industri, maka diperlukan suatu bentuk kerja sama antara pihak perguruan tinggi dengan praktisi industri.
\vspace{0.5ex}

Salah satu bentuk kerja sama yang nyata adalah dengan pelaksanaan Kerja praktik seperti yang tercantum dalam kurikulum di perguruan tinggi, dalam hal ini Institut Teknologi Sepuluh Nopember Surabaya, di lingkungan perusahaan yang menerapkan teknologi yang sesuai dengan bidang studi mahasiswa.
Mata kuliah Kerja Praktik diharapkan dapat mendorong mahasiswa untuk mengenal kondisi di lapangan kerja dan untuk melihat keselarasan antara ilmu pengetahuan yang diperoleh di bangku kuliah dengan aplikasi praktis di dunia kerja.
\vspace{0.5ex}

PT. Aneka Tuna Indonesia sebagai salah satu perusahaan yang bergerak di bidang produksi ikan kaleng di Indonesia, diharapkan dapat menjembatani upaya-upaya perguruan tinggi dalam meningkatkan mutu pendidikannya melalui Kerja praktik sehingga membantu meningkatkan kualitas lulusan perguruan tinggi yang berdaya saing tinggi di dunia industri.
\vspace{0.5ex}

Dalam kerja praktik yang dilakukan, pekerjaan yang dilakukan adalah proses pembuatan \emph{Progressive Web App} Berbasis \emph{Node.js} untuk administrasi loading barang di PT. Aneka Tuna Indonesia.

\newpage

\section{Waktu dan Tempat Pelaksanaan}
\vspace{1ex}

Kerja praktik dilakukan secara Work From Home, dengan dua kali kunjungan di pabrik PT. Aneka Tuna Indonesia, yang beralamant di Jl. Randupitu - Gunung Gangsir No. 36, Asabri, Nogosari, Kec. Pandaan, Pasuruan, Jawa Timur 67156, dari tanggal 6 Juli 2020 sampai dengan 11 September 2020.
\vspace{0.5ex}

\section{Tujuan}
\vspace{1ex}

Secara umum, tujuan dari kerja praktik ini adalah:
\vspace{0.5ex}

\begin{enumerate}[nolistsep]

  \item Menciptakan hubungan yang sinergis, jelas dan terarah antara dunia industri dan perguruan tinggi, dimana output perguruan tinggi merupakan sumber daya manusia dalam dunia industri.
  \vspace{0.5ex}

  \item Meningkatkan kepedulian dan partisipasi oleh dunia industri dalam memberikan kontribusi pada sistem pendidikan nasional.
  \vspace{0.5ex}

  \item Membuka wawasan mahasiswa agar dapat mengetahui dan memahami aplikasi ilmunya di dunia industri.
  \vspace{0.5ex}

  \item Sebagai sarana pembelajaran sosialisasi dalam lingkungan dunia kerja.
  \vspace{0.5ex}

  \item Mahasiswa dapat memahami dan mengetahui sistem kerja di dunia industri sekaligus mampu mengadakan pendekatan masalah yang ada.
  \vspace{0.5ex}

  \item Menumbuhkan dan menciptakan pola berpikir konstruktif yang lebih berwawasan bagi mahasiswa serta meningkatkan kemampuan praktik di bidang pengembangan aplikasi berbasis web dan mengaplikasikan langsung ilmu yang telah dipelajari di Departemen Teknik Komputer FTEIC - ITS.
  \vspace{0.5ex}

  \item Meningkatkan kedisiplinan, kemandirian, dan kepekaan mahasiswa melalui budaya kerja di dalam perusahaan.
  \vspace{0.5ex}

\end{enumerate}
\vspace{0.5ex}

Serta secara khusus, tujuan dari kerja praktik ini adalah:
\vspace{0.5ex}

\begin{enumerate}[nolistsep]

  \item Untuk memenuhi beban satuan kredit semester (SKS) yang harus ditempuh sebagai persyaratan akademis di Departemen Teknik Komputer FTEIC - ITS.
  \vspace{0.5ex}

  \item Mengembangkan pengetahuan, sikap, keterampilan dan kemampuan profesi melalui penerapan ilmu, latihan kerja dan pengamatan teknik yang diterapkan di PT. Aneka Tuna Indonesia.
  \vspace{0.5ex}

  \item Memperdalam pengetahuan mahasiswa dengan mengenal dan mempelajari secara langsung penerapan ilmu teknik komputer.
  \vspace{0.5ex}

  \item Mengembangkan hubungan baik antara pihak perguruan tinggi dengan PT. Aneka Tuna Indonesia.
  \vspace{0.5ex}

  \item Melakukan analisis dan memberikan rekomendasi dalam bentuk laporan kerja praktik kepada PT. Aneka Tuna Indonesia mengenai permasalahan yang dihadapi perusahaan.

\end{enumerate}
\vspace{0.5ex}

\section{Batasan Masalah}
\vspace{1ex}

Dalam penulisan laporan ini akan dibahas tentang pembuatan \emph{Progressive Web App} berbasis \emph{Node.js} untuk administrasi loading barang PT. Aneka Tuna Indonesia.
\vspace{0.5ex}

\section{Metodologi Pengumpulan Data}
\vspace{1ex}

Metodologi yang digunakan pada penyusunan laporan kerja ini adalah:
\vspace{0.5ex}

\begin{enumerate}[nolistsep]

  \item \textbf{Metode Eksperimen}
  \vspace{0.5ex}

  Penulis memperoleh data melalui percobaan langsung pada objek sehingga dapat mengamati pengaruh setiap komponen objek dan hubungan mereka disertai pencatatan tentang pengertian dan fungsi objek dengan singkat dan jelas.
  \vspace{0.5ex}

  \item \textbf{Studi Literatur}
  \vspace{0.5ex}

  Penulis mencatat atau memanfaatkan referensi berupa katalog, arsip-arsip, dan buku-buku. Referensi diperoleh dari perpustakaan dan dokumen perusahaan.
  \vspace{0.5ex}

  \item \textbf{Metode Diskusi}
  \vspace{0.5ex}

  Penulis mengumpulkan data melalui diskusi atau menanyakan secara langsung kepada pembimbing dan pegawai.
  Tujuannya untuk mendapatkan data-data secara langsung dan jelas.
  \vspace{0.5ex}

\end{enumerate}
\vspace{0.5ex}

\section{Sistematika Penulisan}
\vspace{1ex}

Dalam penulisan laporan kerja praktik ini, penulis membagi laporan dalam beberapa bab yang disusun dengan sistematika sebagai berikut:
\vspace{0.5ex}

\begin{enumerate}[nolistsep]

  \item \textbf{Bab I Pendahuluan}
  \vspace{0.5ex}

  Bab ini memaparkan mengenai garis besar kerja praktik yang meliputi latar belakang, waktu dan tempat pelaksanaan, tujuan kerja praktik, batasan masalah, metodologi pengumpulan data, serta sistematika penulisan laporan kerja praktik.
  \vspace{0.5ex}

  \item \textbf{Bab II Profil Perusahaan}
  \vspace{0.5ex}

  Bab ini berisi penjelasan mengenai profil PT. Aneka Tuna Indonesia yang meliputi sejarah, visi dan misi, dan struktur organisasi yang ada di perusahaan tersebut.
  \vspace{0.5ex}

  \item \textbf{Bab III Tinjauan Pustaka}
  \vspace{0.5ex}

  Bab ini berisi penjelasan tentang istilah-istilah atau teori-teori yang digunakan dalam pembuatan kerja praktik dan pustaka yang dipakai.
  \vspace{0.5ex}

  \item \textbf{Bab IV Desain dan Implementasi}
  \vspace{0.5ex}

  Bab ini berisi pemaparan mengenai kebutuhan untuk perancangan beserta implementasi dari sistem yang akan dibangun dan dikembangkan.
  \vspace{0.5ex}

  \item \textbf{Bab V Pengujian dan Evaluasi}
  \vspace{0.5ex}

  Bab ini berisi penjelasan tentang hasil pengujian sistem dan evaluasi yang dilakukan terhadap kinerja sistem secara menyeluruh.
  \vspace{0.5ex}

  \item \textbf{Bab VI Kesimpulan dan Saran}
  \vspace{0.5ex}

  Bab ini berisi kesimpulan dan saran dari proses selama pengerjaan kerja praktik di PT. Aneka Tuna Indonesia.
  \vspace{0.5ex}

\end{enumerate}
\vspace{0.5ex}
